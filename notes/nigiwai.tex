% !TEX TS-program = pdflatexmk
%!TEX encoding = UTF-8 Unicode
 \documentclass[a4paper,12pt]{amsart}
 \usepackage{amsmath,amsthm,amssymb}
 \usepackage{graphicx}
 \usepackage[all]{xy}
 \usepackage{txfonts}
\usepackage{hyperref}
\usepackage[margin=2cm]{geometry}
%\usepackage{enumitem}

\theoremstyle{definition}
\newtheorem{thm}{Theorem}[section]
\newtheorem{lem}[thm]{Lemma}
\newtheorem{cor}[thm]{Corollary}
\newtheorem{prop}[thm]{Proposition}
\newtheorem{dfn}[thm]{\rm Definition}
\newtheorem{rem}[thm]{\rm Remark}
\newtheorem{ex}[thm]{\rm Example}
\numberwithin{equation}{section}

\newcommand{\R}{\mathbb{R}}
\newcommand{\Q}{\mathbb{Q}}
\newcommand{\N}{\mathbb{N}}
\newcommand{\Z}{\mathbb{Z}}
\newcommand{\C}{\mathbb{C}}
\renewcommand{\H}{\mathbb{H}}
\newcommand{\F}{\mathbb{F}}

\newcommand{\red}[1]{{\color{red} #1}}
\newcommand{\blue}[1]{{\color{blue} #1}}

\title{Nigiwai discussion memo}
\date{\today}

\begin{document}
\maketitle

TODO:
\begin{enumerate}
\item (Done!) use synthetic data generated by Crowd Simulation (e.g., \cite{vedere}:
``Scenarios''\footnote{\url{https://gitlab.lrz.de/vadere/vadere/-/tree/master/}} contains demo files.)
\begin{enumerate}
\item We want to have synthesised trajectory data. This can be used to evaluate our indicator definitions.
\item Also, we can use Cinema4D or other rendering software to synthesise videos to test the tracking algorithms.
\end{enumerate}
\item Make distinguishing pairs of synthetic scenarios:
\begin{enumerate}
\item to distinguish absolute and signed velocities:
(1) people going in the same direction (2) opposite direction
\item 
discernible by Nigiwai score but not by a simple indicator like density:
(1) one with just source and sink and people travel smoothly from source to sink
(2) the other with one shop in the middle and people stop at there for some time.
\end{enumerate}
\item define ``ground truth'' to compare with the proposed indicators:
the ultimate goal is to find correlation among different indicators.
\begin{itemize}
\item inquiry, sales (for real data)
\item global and rough indicator: stay time etc. (for real and synthetic data)
\end{itemize}
\item Find candidate journals to which we send our paper, and study the way how people publish there.
(the manner of theory, experiments, writing etc.)
A search ``pedestrian'' in Google Scholar gives several journals including Physical Review E.
\end{enumerate}


%%%%%%%%%
\section{A smoothed variant of the Nigiwai indicator*}

It is highly possible that I am making silly mistakes here.
Just ask if you find anything strange!
Perhaps, we can use Teams chat for that?

\medskip

Setting:
\begin{itemize}
\item $i,j \in \{0,1,\ldots \}$: person ID
\item $t\in \{0,1,\ldots \}$: time (frame)
\item $x_{i,t} \in \R^2$: the coordinates of people (or point-of-interest such as stores).
\item $d_{i,j,t} = |x_{i,t}-x_{j,t}|$: (Euclidean) distance between $i$ and $j$
\item $\tilde{d}_{i,j,t}$: the moving average of $d_{i,j,t}$. For example
if $x_i,x_j$ are both present at time $t-1$, 
\[
 \tilde{d}_{i,j,t} = \alpha d_{i,j,t} + (1-\alpha) \tilde{d}_{i,j,t-1} \ (\text{for example, } \alpha=\frac{7}{8})
\]
otherwise
\[
 \tilde{d}_{i,j,t} = {d}_{i,j,t}.
\]
The value of $\alpha$ has to be chosen according to the frame rate and the strength of noise.
\item $\Delta \tilde{d}_{i,j,t}$: the signed relative velocity
\[
 \Delta \tilde{d}_{i,j,t} = \tilde{d}_{i,j,t} - \tilde{d}_{i,j,t-1}
\]
\item Nigiwai indicator for $i$ is defined to be
\[
 S_i(t) = \sum_{j} \dfrac{\exp(-\Delta \tilde{d}_{i,j,t}/W_v)}{(\tilde{d}_{i,j,t}+C)^2},
\]
where $W_v, C$ are hyper-parameters.
An alternative is
\[
 S_i(t) = \sum_{j} \dfrac{\exp(-\Delta \tilde{d}_{i,j,t}/W_v)}{\min(\tilde{d}_{i,j,t},C)^2},
\]
\end{itemize}

Remarks:
\begin{itemize}
\item I opted for the moving average to reduce the effect of noise. Std can be a better choice as Mohamed did.
\item I guess relative velocity is better; a group of people moving together make Nigiwai!
\item ``Signed'' relative velocity accounts for the fact that someone getting closer adds Nigiwai.
\item The form of the function, e.g., $\exp$ may be altered (to e.g., cubic function $y^3$).
 We have to do trial-and-error to find the optimal one.
\item We can regard a place (e.g., store) as a stationary person.
For a region, we can place multiple ``stationary persons'' to represent the region:
The region's Nigiwai can be defined as the sum of their Nigiwai.
\item Taking the sum $\sum_{j}$ is too much computation.
In practice, we can just take the sum over neighbours:
$\sum_{ \{j \mid d_{i,j,t}< \text{threshold} \}}$
or $\sum_{ \{j \mid j \text{ runs through $k$-nearest neighbours of $i$} \}}$.
$k$-nearest neighbours can be computed by algorithms such as \emph{kd-tree} or \emph{random projections}.
\item Note that Nigiwai indicator $S_i(t)$ is for each person just the same as the Mohamed's version (both for region and for person).
\item To define a global indicator for the whole shopping street, 
we need a way to aggregate local indicators.
The easiest way is to take the sum: 
\begin{enumerate}
\item Business Nigiwai at the moment $t$ is $\sum_{i\mid i \text{runs through shops}} S_i(t)$
\item Visitor Nigiwai at the moment $t$ is $\sum_{i\mid i \text{runs through people}} S_i(t)$
\end{enumerate}
Note that considering local indicator first has several advantages:
\begin{enumerate}
\item to break up the problem into smaller parts (modularization):
\item to construct different global indicators out of local ones like the above
\item local indicators can be used for, e.g., visualisation in the form of heatmap.
\end{enumerate}

\end{itemize}


\begin{thebibliography}{99}
\bibitem{vedere} Vedere,
\url{https://arxiv.org/pdf/1907.09520.pdf}, 
\url{http://www.vadere.org/getting-started/}
 
\end{thebibliography} 

\end{document}
